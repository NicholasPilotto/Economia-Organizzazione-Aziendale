\subsection{Ammortamento}
L'ammortamento è un procedimento amministrativo-contabile con cui il costo di un bene viene ripartito nel corso di più esercizi.

Oggetto del procedimento di ammortamento sono i cosiddetti beni a fecondità ripetuta, ovvero, che mantengono la loro utilità nel corso del tempo, attraverso la procedura di ammortamento, infatti, il costo di tali beni viene spalmato su più anni in ragione della loro durata economica.

La decisione da parte di un’azienda di ripartire il costo di un bene su più anni viene messa in pratica suddividendo il costo del bene in più quote, il cui numero varia in funzione del numero di esercizi in cui il bene (impianto, macchinario etc.) sarà utilizzato.

Ad imporre l’ammortamento è anche il principio contabile della competenza economica delle componenti reddituali, secondo cui non è possibile imputare un bene che viene utilizzato in più esercizi interamente all'esercizio in cui è stato acquistato.
Oggetto dell'ammortamento possono essere:
\begin{itemize}
	\item Le immobilizzazioni materiali ovvero l’insieme di tutti i fattori produttivi ad utilità pluriennale fisicamente tangibili (es. fabbricati, macchinari, impianti, automezzi, attrezzature industriali e commerciali, computer, mobili d'ufficio ecc.)
	\item Le immobilizzazioni immateriali come l’insieme di tutti i fattori produttivi ad utilità pluriennale non fisicamente tangibili (ad esempio, brevetti e marchi, diritti di utilizzo di opere dell'ingegno, concessioni governative, costi di ricerca e sviluppo, costi di pubblicità ecc.)
\end{itemize}
Mentre per le immobilizzazioni materiali viene usato spesso il metodo indiretto, che fa confluire ogni anno la quota nel fondo ammortamento; per le immobilizzazioni immateriali si applica il metodo diretto, consistente nel portare direttamente in deduzione dal costo storico del bene pluriennale le quote d'ammortamento

\begin{itemize}
	\item \textbf{Accantonamento:} A$_i$ = quota annua di ammortamento $\rightarrow$ C.E.
	\item \textbf{Fondo di ammortamento:} $\sum$A$_i$ = somma delle quote accantonate $\rightarrow$ Nota integrativa
	\item \textbf{Valore storico:} V$_0$ = valore di acquisto del bene $\rightarrow$ Nota integrativa
	\item \textbf{Valore netto contabile:} V$_0$ - $\sum$A$_i$ = stima del valore attuale del bene $\rightarrow$ C.E.
\end{itemize}

Alla vendita del bene si hanno i seguenti scenari:
\begin{itemize}
	\item Se Valore di mercato = Valore netto contabile $\rightarrow$ S.P. attivo (Cassa): valore macchinario
	\item Se Valore di mercato > Valore netto contabile $\rightarrow$ C.E. (Plusvalenza=ricavo/provente): V$_m$-V$_{nc}$
	\item Se Valore di mercato < Valore netto contabile $\rightarrow$ C.E. (Minusvalenza=costo): V$_m$-V$_{nc}$
\end{itemize}