\section{Il Bilancio}
Questa parte del Corso ci illustra come un'azienda genera reddito, il proprio modello finanziario e i suoi aspetti finanziari.

I contenuti che verranno presentati sono:
\begin{itemize}
	\item Introduzione al bilancio e concetto di competenza
	\item Bilancio giuridico: stato patrimoniale e conto economico
	\item Riclassificazione di bilancio
	\item Analisi di bilancio
\end{itemize}

\subsection{Come un'azienda genera reddito}
Il reddito di impresa è il profitto economico realizzato dall’esercizio di un’attività economica e richiede l’impiego di risorse umane, di mezzi, di attrezzature e di tutto ciò che occorre per produrre beni o servizi.

Il reddito d’impresa è la differenza tra le componenti positive e le componenti negative del reddito stesso.

Le componenti positive di reddito sono:
\begin{itemize}
	\item ricavi: realizzati dalle vendite di beni o servizi durante l’esercizio
	\item plusvalenze patrimoniali: vendita di un macchinario usato ad un valore superiore al suo valore contabile
	\item sopravvenienze attive: rinuncia ad incassare un credito da parte di un fornitore
	\item proventi finanziari: interessi attivi maturati sui conti correnti bancari o postali, o sui crediti verso clienti o soggetti diversi
	\item rivalutazioni: di immobili, di quote azionarie, ecc...
	\item variazione positiva delle rimanenze finali di merci, prodotti finiti, semilavorati, materie prime, rispetto alle loro esistenze iniziali
\end{itemize}

Mentre le componenti negative di reddito sono:
\begin{itemize}
	\item costi: costi di acquisto delle merci, costi del personale ecc...
	\item minusvalenze patrimoniali: vendita di un impianto usato ad un valore inferiore al suo valore contabile
	\item sopravvenienze passive: una multa, un risarcimento a terzi
	\item oneri finanziari: interessi passivi che maturano su debiti verso le banche, i fornitori, ecc...
	\item ammortamenti: quota del costo d’acquisto di alcuni beni aziendali ad utilità pluriennale che si fa incidere sul reddito dell’esercizio
	\item accantonamenti: quote di costi che si fanno pesare sul reddito d’esercizio in previsione di eventi futuri (es. quota fondo \textit{TFR})
	\item svalutazioni: di immobili, di quote azionarie, ecc...
	\item imposte: alcuni tipi di imposte correnti e differite, in misura totale o parziale
\end{itemize}

\begin{center}

\begin{figure}[H]
	\includegraphics[width=0.8\linewidth]{resources/chapters/Bilancio/images/vita-impresa.png}
	\centering
	\caption{Modello rappresentate la vita di un'impresa}
\end{figure}

\begin{figure}[H]
	\includegraphics[width=0.8\linewidth]{resources/chapters/Bilancio/images/modello-economico-finanziario.png}
	\centering
	\caption{Modello rappresentate il modello Economico-Finanziario di un'impresa}
\end{figure}
\end{center}


\subsection{Aspetti economici e patrimoniali dello scambio}
\begin{table}[h!]
	\begin{tabular}{|c | c | c|}
		\hline
		& Cessione di beni/servizi & Acquisizione di beni/servizi \\
		\hline
		Aspetto economico & Ricavo & Costo \\
		\hline
		Aspetto patrimoniale & & \\
		- finanziario & Credito & Debito \\
		- economico & Entrata di moneta & Uscita di moneta \\
		\hline
	\end{tabular}
	\centering
	\caption{Schematizzazione modello economico-finanziario}
\end{table}

\subsection{Capitale di Richio e di Prestito}
