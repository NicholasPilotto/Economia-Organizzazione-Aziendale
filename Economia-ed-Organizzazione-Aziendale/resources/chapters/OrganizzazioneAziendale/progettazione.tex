\subsection{Progettazione}
La progettazione nello sviluppo di prodotti ricerca la migliore risposta, in termini di prodotto, alle aspettative del cliente. I principali compiti sono:
\begin{itemize}
	\item Definisce la forma del prodotto
	\item Identifica le prestazioni
	\item Disegna i componenti
	\item Ricerca la migliore soluzione “producibile”
	\item Valuta la “funzione” dei componenti
	\item Garantisce la migliore prestazione (globale), entro i limiti di costo assegnati
\end{itemize}

La sviluppo di nuovi prodotti si suddivide in vari step. Fondamentale è partire da un'idea, per poi:
\begin{enumerate}
	\item \textbf{Analisi e Pianificazione preliminare}: Analisi preliminare tecnica, economica, di mercato. Primo piano di prodotto.
	\item \textbf{Concetto e pre-desing}: Bisogni dei clienti, analisi aspetti tecnici ed economici, architettura preliminare.
	\item \textbf{Sviluppo tecnico}: Progettazione di dettaglio, progettazione processi produttivi, primi prototipi.
	\item \textbf{Testing e prove di collaudo}: Test presso i clienti, pianificazione processo produttivo e lancio.
	\item \textbf{Lancio}: Organizzazione lancio, preparazione documentazione di prodotto, comunicazione.
\end{enumerate}
